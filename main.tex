no%--------------------------------------------------------------
%	ARCHIVO PRINCIPAL
%--------------------------------------------------------------

%	Tipo de documento
\documentclass[12pt]{report}
\setcounter{secnumdepth}{2}
%	Lista de paquetes a utilizar
\usepackage[utf8]{inputenc}
\usepackage{graphicx}
\usepackage[spanish]{babel}
\usepackage[fixlanguage]{babelbib}
\selectbiblanguage{spanish}
\usepackage{url}
\usepackage{tocloft}
\usepackage[right=3cm,left=3cm,top=2cm,bottom=2cm,headsep=0.5cm,footskip=0.5cm]{geometry}
\usepackage{fancyhdr}
\usepackage{emptypage}
%\usepackage[numbers]{natbib}
\usepackage{cite}
\usepackage[nottoc,notlot,notlof]{tocbibind}
\usepackage{flafter}
\usepackage{float}
%	Ubicación de las imágenes 
\graphicspath{ {Figuras/} }


\usepackage{caption}
\newcommand{\source}[1]{\vspace{-3pt} \caption*{\emph{\centering{ Fuente: {#1}}}} }


% 	Aquí definimos el encabezado de las paginas 
\lhead[L]{}
\chead[]{}
\rhead[]{\thepage}
\renewcommand{\headrulewidth}{0.5pt}


% % 	Aquí definimos el pie de pagina de las paginas
% \lfoot[]{}
% \cfoot[]{}
% \rfoot[]{Titulo del documento}
% \renewcommand{\footrulewidth}{0.5pt}

% 	Aquí definimos el encabezado y pie de pagina de la pagina inicial de un capitulo.
\fancypagestyle{plain}{
%	Encabezado
\fancyhead[L]{\rightmark}
\fancyhead[C]{}
\fancyhead[R]{\thepage}
%	Pie de pagina
% \fancyfoot[L]{}
% \fancyfoot[C]{}
% \fancyfoot[R]{Titulo del documento}
\renewcommand{\headrulewidth}{0.5pt}
% \renewcommand{\footrulewidth}{0.5pt}
}
\setlength{\headheight}{16pt}

%	Estilo de las paginas (Activa encabezados y pie de pagina)
\pagestyle{fancy}


%	Tipo de referencias
\bibliographystyle{babplain}

%	Titulo del documento
\title{Plantilla de documentos universitarios}
%	Autor del documento
\author{Mauricio Andres Machuca Pavan}

%***************** Estructura del documento *****************
\setlength{\parskip}{1em}
\begin{document}

%	Portada
\label{ch:portada}
\thispagestyle{empty}

\begin{figure}[!htb]
	\minipage{0.15\textwidth}
	\raggedright
	\includegraphics[width=\linewidth]{Figuras/ucv_logo.jpg}
	\endminipage\hfill
	\minipage{0.15\textwidth}
	\raggedleft
	\includegraphics[width=\linewidth]{Figuras/logo_ciencias.jpg}
	\endminipage\hfill
\end{figure}

\begin{center}
	Universidad Central de Venezuela\\
	Facultad de Ciencias\\
	Escuela de Computación\\
	
\end{center}

\vspace{2.5cm}
\begin{center}
	\large{\textbf{ Sistema de conversión e integración de programas automatizados de chat para la plataforma Zendesk Chat.}}
\end{center}

\vspace{5.0cm}
\begin{center}
	Trabajo de Seminario \\
	presentado ante la Ilustre\\
	Universidad Central de Venezuela\\
	Por el Bachiller\\
	Mauricio Andres Machuca Pavan\\
\end{center}

\begin{center}
	Tutor:\\ Prof. Franklin Sandoval\\
\end{center}

\vspace{1.0cm}
\begin{center}
	Caracas, Noviembre 2019
\end{center}
\pagenumbering{Roman} 
\chapter*{Introducción}
\addcontentsline{toc}{chapter}{Introducción}


%	Indice / Tabla de Contenidos
%--------------------------------------------------------------
%	Indice de contenidos
%--------------------------------------------------------------
\renewcommand{\cftchappresnum}{Capítulo }
\renewcommand{\cftchapaftersnum}{:}
\renewcommand{\cftchapnumwidth}{6em}
\tableofcontents

%--------------------------------------------------------------
%	Indice de Figuras
%--------------------------------------------------------------
\newpage
\listoffigures


\pagenumbering{arabic} 

%	Capitulo uno
\chapter{Problema de Investigación}
\markboth{Problema de Investigación}{Problema de Investigación}

\section{Título}
Desarrollo de solución para integrar la plataforma ``Zendesk Chat''  con el sistema de automatización de conversaciones ``Oracle Digital Assistant''.

\section{Planteamiento del problema}
% cual es el problema? Se necesita automatizar la atencion al cliente via chat pero permites un grado de atención personal de ser necesaria
Zendesk Chat originalmente Zopim, es un producto de software web adquirido por la compañia Zendesk para ofrecer la posibilidad de chat en vivo entre su suite de soluciones en el área de atención al cliente. Debido al éxito que ha tenido Zendesk, el uso de sus productos se ha expandido en el área de soporte, siendo Zendesk Chat la opcion de preferencia para aquellos entes que necesitan ofrecer una forma de interacción directa con sus clientes.

Sin embargo esta solución no es perfecta, entre los defectos podemos nombrar dos que influyen en su implementación por aquellos interesados en el entorno Zendesk. En primer lugar por el volumen de interacciones que pueden generarse en las plataformas en linea, el coste que requeriría un equipo de personal capacitado para la atención al cliente manteniendo un nivel de respuesta a la par de un servicio que busca la interactividad de una plataforma de conversación social es alto, y el segundo punto a considerar como defecto es que una gran parte de las interacciones de soporte de cualquier área son de carácter repetitivo, lo que implica que tareas fácilmente automatizables son llevadas a cabo por personal entrenado sumando un costo extra a la implementación de este producto.

Gracias a que la plataforma Zendesk ofrece interacción vía API para la mayoría de sus productos incluyendo Zendesk Chat, se plantea desarrollar un sistema que conecte Oracle Digital Assistant, la cual es la plataforma de Oracle para la creacion de bots y automatización de interacciones, con Zendesk Chat, ofreciendo así la posibilidad de automatizar interacciones además de permitir a los usuarios de la plataforma Zendesk Chat la opción de habilitar para sus clientes una vía de atención directa en vivo si el sistema automatizado no cumple con sus necesidades.


\section{Justificación}
Los servicios de soporte y atención al cliente son utilizadas como apoyo para un amplio abanico de soluciones. Su uso es esencial para mantener un nivel de calidad y aceptación del producto ante los usuarios del mismo, esto es de especial importancia para plataformas con grandes volúmenes de usuarios, donde la atención se lleva a cabo por medio de aplicaciones similares a plataformas de conversación social, donde el usuario se siente atendido de forma directa. 

La implementación de un sistema automatizado de conversaciones dentro del área de soporte y atención al cliente busca precisamente mejorar el funcionamiento de estas soluciones, al permitir que la atención personalizada se centre en los casos que requieren un nivel de resolución mas complejo, sin sacrificar la sensación de atención personal que aporta una plataforma de conversaciones social.


\section{Objetivo General}

Desarrollar software que permita integrar el sistema de automatización de conversaciones de Oracle Digital Assistant con la plataforma de soporte Zendesk Chat conservando la capacidad de atención al cliente por personal humano.

    
\section{Objetivos Específicos}
    \begin{itemize}
        \item Configurar Oracle Digital Assistant y Zendesk Chat con las herramientas web que ofrecen para permitir integraciones entre aplicaciones de terceros.
        
        \item Implementar un servidor que maneja las solicitudes y respuestas de las plataformas Oracle Digital Assistant y Zendesk Chat.
        
       	\item Desarrollar los algoritmos que traduzcan la estructura de solicitudes y respuestas de los servicios Oracle Digital Assistant a Zendesk Chat y viceversa.
    
    	\item Añadir a la solución desarrollada la capacidad de la operación manual por un agente cuando el usuario lo requiera.
                
        \item Programar la automatización de una conversación en Oracle Digital Assistant para probar el funcionamiento de la integración
        
    \end{itemize}
    
    
    
\section{Solución Propuesta}

Se propone el desarrollo de un servidor que maneje solicitudes de las plataformas Oracle Digital Assistant (ODA) y Zendesk Chat, de tal forma que se logre integrar un sistemas automatizado de conversaciones creado en la plataformas ODA con la plataforma de soporte y atención al cliente Zendesk Chat.

Dicha solución permitirá que tareas repetitivas sean atendidas por el sistema automatizado, manteniendo la interacción con el usuario a travez de la interfaz de chat de la plataforma Zendesk Chat, además se implementara la capacidad para manejar el control del flujo de la conversación entre el sistema automatizado y un agente humano.

    \subsection{Arquitectura conceptual a utilizar}
    Se plantea implementar un servidor web que maneje peticiones de las  plataformas Oracle Digital Assistant y Zendesk Chat, y emita respuestas adecuadas en el formato que dichas plataformas aceptan.
    
    La arquitectura de la solución se puede dividir en dos casos. El primer caso  se distingue por la interacción de ODA con el servidor web, este proceso se demuestra en la figura \ref{fig:arqconoda}, el usuario interactúa con la interfaz de chat administrada por Zendesk Chat, los mensajes del usuario serán transmitidos por Zendesk vía websockets al servidor web, donde serán adaptados para su envió a la plataforma de Oracle. La plataforma de ODA procesara el mensaje y enviara un mensaje de respuesta al servidor web de acuerdo a la configuración del bot, dicho mensaje sera luego transmitido a la plataforma de Zendesk.
    \begin{figure}[htpb]
        \centering
        \includegraphics[width=0.85\textwidth]{Figuras/propuestasolucion1.png}
        \caption{Arquitectura conceptual, interacción con ODA}
        \label{fig:arqconoda}
    \end{figure}
    
    En la figura \ref{fig:arqconagente} se presenta el segundo caso; el servidor web recibe una petición de interacción con un agente humano por parte del usuario, el servidor web se encarga de enviar el control de la conversación a Zendesk para que el agente mantenga una conversación directa con el usuario, una vez el agente decida culminar la conversación el servidor web retornara el control al bot.
    
    \begin{figure}[htpb]
        \centering
        \includegraphics[width=0.85\textwidth]{Figuras/propuestasolucion2.png}
        \caption{Arquitectura conceptual, interacción con Agente}
        \label{fig:arqconagente}
    \end{figure}
    
    
    \subsection{Arquitectura tecnológica a utilizar}
    %Nodejs express librerias oracle zendesk chat
    Para la implementación de la solución se utilizara Node.js para el desarrollo del servidor web y Express.js como el framework principal, esto permitirá hacer uso de las cualidades REST que ofrece Express.js manteniendo un nivel de consistencia y estandarización con la arquitectura REST. Para la interacción con las plataformas de terceros se incluirán las librerías aportadas por Oracle Digital Assistant y Zendesk Chat, estas librerías son desarrolladas en Javascript y están soportadas por Node.js. 
    
    El servidor sera estructurado siguiendo un modelo de microservicios, lo que permitirá un enfoque en la alta mantenibilidad de la solución y facilidad para su implementación y despliegue por un equipo pequeño.
    
    Los datos generados por la aplicación son generados por las plataformas de terceros por lo que la solución no necesita hacer uso de una tecnología de base de datos para su implementación, toda información sera almacenada por Oracle o Zendesk utilizando la tecnología por defectos de sus sistemas.
    
    \subsection{Metodología de desarrollo a utilizar}
    %SCRUM
    %Para guiar el desarrollo de este proyecto se propone el uso de la metodología de desarrollo ágil SCRUM, por su facilidad de adaptación diseñada para ofrecer un valor significativo de forma rápida en todo proyecto. La misma permite presentar soluciones parciales mediante el uso de iteraciones, lo que hará que existan pruebas que contribuyan con el progreso del desarrollo de la aplicación al poder obtener retroalimentación de parte del cliente. 
    
    El desarrollo de este proyecto se llevara bajo un marco metodológico ágil basado en SCRUM, haciendo uso del backlog, los sprint, y la propiedad iterativa del mismo. Sin embargo este proyecto sera llevado por un único individuo por tanto las propiedades del modelo SCRUM aplicadas a equipos no estarán presente como parte de la metodología en este proyecto. 
    
    % \subsection{Descripción del flujo asociado a la solución}
    

\section{Alcance}
    \begin{itemize}
        \item Generar los permisos adecuados en la plataforma Zendesk Chat para integrar por medio de su API REST servicios de terceros.
        
        \item Implementar un servidor en Node.js usando Express.js como framework, instalando en el mismo mediante NPM las librerías ofrecidas por Oracle y Zendesk para el uso con sus productos.
        
        \item Desarrollar un servicio que maneje los mensajes provenientes de Oracle Digital Assistant y traduzca su estructura para ser comprendidos por la plataforma  Zendesk Chat
        
        \item Desarrollo de algoritmo que maneje el uso de mensajes con formatos especiales que incluyan botones o respuestas rápidas entre Zendesk Chat y Oracle Digital Assistant
        
        \item Implementar la capacidad de cambiar el control del flujo de la conversación entre el sistema automatizado y un agente humano.
        
        \item Manejo de errores de conexión y reactivación automática de la conexión por websocket entre Zendesk Chat y el servidor.
        
        \item Configurar un Canal (Channel) de Oracle Digital Assistant, que permita la integración por medio de webhooks y su API REST de aplicaciones de terceros con el servicio ofrecido por Oracle.
                
        \item Crear un sistema de respuestas automáticas usando el producto `Habilidades' (Skills) de Oracle Digital Assistant.
        
        \item Realizar el proceso de integración del sistema de respuestas automatizadas mediante el canal de Oracle Digital Assistant para comprobar el funcionamiento de la integración con Zendesk Chat.
        
        \item Simular el proceso de cambio de control de flujo entre el sistema automatizado y un agente humano en Zendesk Chat.
        
    \end{itemize}
\chapter{Marco Aplicativo}
\markboth{Marco Aplicativo}{Marco Aplicativo}

\section{Descripción general de la solución}

\section{Aplicación de la metodología}

\section{Requerimientos del sistema}

\subsection{Requerimientos funcionales}

\subsection{Requerimientos no funcionales}

\section{Descripción del flujo asociado a la solución}

% \section{Análisis del modelo de datos}

\section{Descripción de los módulos del sistema e interfaces}

\section{Fase de pruebas}
% \include{Capitulos/MarcoTeorico}

% \chapter{Marco Metodológico}
\markboth{Marco Metodológico}{Marco Metodológico}

\section{Metodologia Fundacional para la Ciencia de Datos}
    \lhead[\thepage]{\thesection Metodologia Fundacional para la Ciencia de Datos}

    Según \cite{foundationalmet} en el ámbito de la ciencia de datos, una metodología es una estrategia general que guía los procesos y actividades en un dominio dado. La metodología no depende de herramientas o tecnologías en particular, ni es un conjunto de técnicas o récipes. Mas bien, una metodología provee al científico de datos con un marco de trabajo de como proceder con cualquier método, procesos y heurísticas que sean usadas para obtener respuestas o resultados.


    Resolver problemas y responder preguntas a través de análisis de datos es una practica estandarizada en ciencia de datos, como meta principal del área esta el obtener información del proceso de análisis, que luego es utilizada por las organizaciones para tomar alguna acción requerida por la misma.\cite{foundationalmet}

    La Metodologia Fundacional para la Ciencia de Datos propuesta en \cite{foundationalmet} busca establecer un marco de trabajo que cumpla con la definición de metodología del autor y provea una guia de estrategias independiente de la tecnología, volumen de los datos o abordaje.


    \begin{figure}[htpb]
        \centering
        \includegraphics[width=0.85\textwidth]{Figuras/foundational_met_image.png}
        \caption{Metodologia Fundacional para la Ciencia de Datos}
        \label{fig:foundmetfig}
    \end{figure}

    En la Figura \ref{fig:foundmetfig} se muestra la metodología de 10 fases propuesta en \cite{foundationalmet}. Este marco de trabajo posee algunas similitudes con metodologías reconocidas para minería de datos, pero enfatiza varias de las nuevas practicas en ciencia de datos como el uso de grandes volúmenes de datos, la incorporación de análisis de texto en modelos predictivos y la automatización de algunos procesos.

    \subsection{La 10 fases de la Metodologia Fundacional para la Ciencia de Datos}
    \begin{enumerate}
        \item Entendimiento del negocio (\emph{Business understanding}): Sin importar el tamaño o alcance de un proyecto, este debe comenzar por la comprensión del negocio, de dicho entendimiento dependerá en gran medida el éxito de la solución al problema. Es vital para los representantes del negocio que la solución analítica sea fundamental en esta etapa mediante la definición del problema, los objetivos del proyecto y los requerimientos de la solución.
        \item Enfoque analítico (\emph{Analytic approach}): Una ves definido el problema el científico de datos estable el enfoque analítico. Para ello el problema es expresado en técnicas de \emph{machine learning} para que  el científico de datos logre identificar cuáles son las técnicas que mejor se adaptan para conseguir los resultados deseados.
        \item Requerimientos de datos (\emph{Data requirements}): La elección del enfoque analítico determina los requerimientos de datos. Los métodos analíticos a usar requieren contenido, formato y representaciones particulares de los datos.
        \item Recolección de datos (\emph{Data collection}): Las fuentes de datos (estructurados, no estructurados, semi-estructurado) relevantes para el dominio del problema son identificadas y agrupadas por el científico de datos. En caso de encontrar brechas en la recolección de datos, podría ser necesario revisar los requerimientos de datos y recolectar más datos.
        \item Entendimiento de los datos (\emph{Data understanding}): El científico de datos debe entender el contenido de los datos, evaluar su calidad y descubrir características de interés para ello se puede apoyar en técnicas de visualización y estadísticas descriptivas.
        \item Preparación de los datos (\emph{Data preparation}): Actividades que incluyen la limpieza de datos, combinación de múltiples fuentes y transformación de datos en variables más prácticas, que preparan los datos para ser usados en la etapa de  modelado.
        \item Modelado (\emph{Modeling}): A partir de una versión inicial del conjunto de datos preparado, el científico de datos utiliza conjuntos de entrenamiento (datos históricos de los cuales los resultados son conocidos), para desarrollar modelos predictivos o descriptivos utilizando el enfoque analítico anteriormente descrito. Esta etapa es altamente iterativa.
        \item Evaluación (\emph{Evaluation}): La evaluación de la calidad del modelo implica verificar si este
        dirige el problema de forma completa y apropiada. Esto requiere que el científico de datos aplique distintos diagnósticos y medidas computadas utilizando el conjunto de entrenamiento y el modelo predictivo.
        \item Despliegue (\emph{Deployment}): Con un modelo satisfactorio, aprobado por los representantes del negocio, se procede al despliega dentro del ambiente de producción u otro ambiente de prueba equiparable, para poder obtener una evaluación de su rendimiento.
        \item Retroalimentación (\emph{Feedback}): Recolectando los resultados obtenidos luego de la implementación de la fase de despliegue, la organización obtiene datos sobre el rendimiento del modelo y observa como el mismo afecta el ambiente de producción. Analizar dicha dicha información permite a los científicos de datos refinar el modelo, incrementar su precisión y con esto su utilidad.
    \end{enumerate}

\section{Modelo CRISP-DM}
\lhead[\thepage]{\thesection Modelo CRISP-DM}

CRISP-DM concebido en 1996, es el acrónimo en ingles de \textit{Cross Industry Standard Process for Data Mining}. Fue creado con la meta de ser una metodología estándar, neutral a industria, herramienta o aplicación.\cite{chapman2000crisp}


En \cite{chapman2000crisp} esta metodología es descrita en términos de un modelo de procesos jerárquicos, consiste de cuatro niveles tal como se muestran en la Figura \ref{fig:crispniveles}.

\begin{figure}[H]
    \centering
    \includegraphics[width=0.7\textwidth]{Figuras/niveles_crisp.png}
    \caption{Los 4 niveles de la metodología CRISP-DM}
     \source{http://crisp-dm.eu/wp-content/uploads/2013/03/Four-Level-Breakdown-of-the-CRISP-DM-Methodology.jpg}
    \label{fig:crispniveles}
\end{figure} 

El primer nivel consta de fases, cada una de estas consiste en varias tareas genéricas (\emph{Generic Taks}), dichas tareas intentan ser lo suficientemente generales como para cubrir todas las posibles situaciones de minería de datos. Buscan ser completas, esto es, cubrir tanto los procesos como aplicaciones de la minería de datos. Estables, el modelo debe ser valido para nuevos desarrollos como nuevas técnicas de modelado.\cite{chapman2000crisp}

En el tercer nivel, el nivel de las tareas especializadas(\emph{Specialized Task}) se describen como deben llevarse acabo, bajo ciertas situaciones, las acciones en el segundo nivel. En el cuarto nivel, la instancias de procesos (\emph{Process Instances}), es un registro de acciones, decisiones y resultados de un abordaje real de minería de datos. Una instancia de proyecto es organizada de acuerdo a la tarea definida en los niveles superiores, pero representa lo que realmente pasa en un - particular, que lo que pasa en general.\cite{chapman2000crisp}


\begin{figure}[H]
    \centering
    \includegraphics[width=0.85\textwidth]{Figuras/crip_fases.png}
    \caption{Las 6 fases del proceso de minería de datos}
     \source{http://crisp-dm.eu/wp-content/uploads/2013/07/newcrispdiagram.gif}
    \label{fig:crispfases}
\end{figure} 


La Figura \ref{fig:crispfases} muestra las fases de un proceso de minería de datos según el modelo de referencia de CRISP-DM. La secuencia entre las fases es flexible. Depende del resultado obtenido en cada fase cual fase sera la siguiente o que tarea particular de una fase sera llevada a cabo. Las flechas indican las dependencias mas importantes y frecuentes entre fases. El circulo exterior en la Figura \ref{fig:crispfases} indica la naturaleza ciclica del proceso de minería de datos.\cite{chapman2000crisp}

Siguiendo lo descrito en \cite{chapman2000crisp} cada una de las fases y sus principales tareas son :
\begin{enumerate}
   
 \item Comprensión del negocio (Objetivos y requerimientos desde una perspectiva no técnica)
    \begin{enumerate}
    \item Establecimiento de los objetivos del negocio (Contexto inicial, objetivos, criterios
    de éxito).
    \item Evaluación de la situación (Inventario de recursos, requerimientos, supuestos,
    terminologías propias del negocio).
    \item Establecimiento de los objetivos de la minería de datos (objetivos y criterios de
    éxito).
    \item Generación del plan del proyecto (plan, herramientas, equipo y técnicas).
    \end{enumerate}

 \item  Comprensión de los datos (Familiarizarse con los datos teniendo presente los objetivos
del negocio)
    \begin{enumerate}
    \item Recopilación inicial de datos.
    \item Descripción de los datos.
    \item Exploración de los datos.
    \item Verificación de calidad de datos.
    \end{enumerate}

 \item  Preparación de los datos (Obtener la vista minable o \textit{dataset})
    \begin{enumerate}
    \item Selección de los datos.
    \item Limpieza de datos.
    \item Construcción de datos.
    \item Integración de datos.
    \item Formateo de datos.
    \end{enumerate}

 \item  Modelado (Aplicar las técnicas de minería de datos a las vistas minables)
    \begin{enumerate}
    \item Selección de la técnica de modelado.
    \item Diseño de la evaluación.
    \item Construcción del modelo.
    \item Evaluación del modelo.
    \end{enumerate}

 \item Evaluación (De los modelos de las fases anteriores para determinar si son útiles a las
necesidades del negocio)
    \begin{enumerate}
        \item Evaluación de resultados.
        \item Revisar el proceso.
        \item Establecimiento de los siguientes pasos o acciones.
    \end{enumerate}

 \item  Despliegue (Explotar utilidad de los modelos, integrándolos en las tareas de toma de
decisiones de la organización)
    \begin{enumerate}
       \item Planificación de despliegue.
    \item Planificación de la monitorización y del mantenimiento.
    \item Generación de informe final.
    \item Revisión del proyecto.
    \end{enumerate}


\end{enumerate}


%	Referencias bibliográfica
\bibliography{Referencias/Referencias}
\end{document}
